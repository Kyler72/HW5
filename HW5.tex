% Options for packages loaded elsewhere
\PassOptionsToPackage{unicode}{hyperref}
\PassOptionsToPackage{hyphens}{url}
\documentclass[
]{article}
\usepackage{xcolor}
\usepackage[margin=1in]{geometry}
\usepackage{amsmath,amssymb}
\setcounter{secnumdepth}{-\maxdimen} % remove section numbering
\usepackage{iftex}
\ifPDFTeX
  \usepackage[T1]{fontenc}
  \usepackage[utf8]{inputenc}
  \usepackage{textcomp} % provide euro and other symbols
\else % if luatex or xetex
  \usepackage{unicode-math} % this also loads fontspec
  \defaultfontfeatures{Scale=MatchLowercase}
  \defaultfontfeatures[\rmfamily]{Ligatures=TeX,Scale=1}
\fi
\usepackage{lmodern}
\ifPDFTeX\else
  % xetex/luatex font selection
\fi
% Use upquote if available, for straight quotes in verbatim environments
\IfFileExists{upquote.sty}{\usepackage{upquote}}{}
\IfFileExists{microtype.sty}{% use microtype if available
  \usepackage[]{microtype}
  \UseMicrotypeSet[protrusion]{basicmath} % disable protrusion for tt fonts
}{}
\makeatletter
\@ifundefined{KOMAClassName}{% if non-KOMA class
  \IfFileExists{parskip.sty}{%
    \usepackage{parskip}
  }{% else
    \setlength{\parindent}{0pt}
    \setlength{\parskip}{6pt plus 2pt minus 1pt}}
}{% if KOMA class
  \KOMAoptions{parskip=half}}
\makeatother
\usepackage{graphicx}
\makeatletter
\newsavebox\pandoc@box
\newcommand*\pandocbounded[1]{% scales image to fit in text height/width
  \sbox\pandoc@box{#1}%
  \Gscale@div\@tempa{\textheight}{\dimexpr\ht\pandoc@box+\dp\pandoc@box\relax}%
  \Gscale@div\@tempb{\linewidth}{\wd\pandoc@box}%
  \ifdim\@tempb\p@<\@tempa\p@\let\@tempa\@tempb\fi% select the smaller of both
  \ifdim\@tempa\p@<\p@\scalebox{\@tempa}{\usebox\pandoc@box}%
  \else\usebox{\pandoc@box}%
  \fi%
}
% Set default figure placement to htbp
\def\fps@figure{htbp}
\makeatother
\setlength{\emergencystretch}{3em} % prevent overfull lines
\providecommand{\tightlist}{%
  \setlength{\itemsep}{0pt}\setlength{\parskip}{0pt}}
\usepackage{bookmark}
\IfFileExists{xurl.sty}{\usepackage{xurl}}{} % add URL line breaks if available
\urlstyle{same}
\hypersetup{
  hidelinks,
  pdfcreator={LaTeX via pandoc}}

\author{}
\date{\vspace{-2.5em}}

\begin{document}

\section{Homicide Mapping in Denver,
Colorado}\label{homicide-mapping-in-denver-colorado}

This assignment visualizes homicide patterns in Denver, Colorado using
the Washington Post homicide data.

I created an \texttt{sf} map showing homicide locations, faceted by
solved status and colored by victim race group.

\subsection{Tasks Completed}\label{tasks-completed}

\begin{itemize}
\tightlist
\item
  Filtered the dataset to include only homicides that occurred in
  Denver, CO.
\item
  Converted homicide locations into an \texttt{sf} object for spatial
  data visualization.
\item
  Downloaded census tract boundaries for Denver using the
  \texttt{tigris} package.
\item
  Created a variable indicating whether each homicide was Solved
  (``Closed by arrest'') or Unsolved (``Closed without arrest'',
  ``Open/No arrest'').
\item
  Grouped victim race into the top three most frequent categories,
  lumping all others into ``Other''.
\item
  Produced a faceted map using \texttt{ggplot2} showing:

  \begin{itemize}
  \tightlist
  \item
    Sub-city geography (tracts)
  \item
    Homicide locations as points
  \item
    Facets for Solved vs Unsolved
  \item
    Colors representing victim race (top 3 categories)
  \end{itemize}
\end{itemize}

\subsection{Files}\label{files}

\begin{itemize}
\tightlist
\item
  \texttt{scripts/} --- Contains the R script used to produce the map.
\item
  \texttt{README.md} --- This document.
\end{itemize}

\subsection{Required R Packages}\label{required-r-packages}

\begin{itemize}
\tightlist
\item
  \texttt{dplyr}
\item
  \texttt{sf}
\item
  \texttt{tigris}
\item
  \texttt{ggplot2}
\item
  \texttt{forcats}
\item
  \texttt{readr}
\end{itemize}

\subsection{Author}\label{author}

Kyle Ruszkowski

\end{document}
